
% a few handy macros

\newcommand\matlab{{\sc matlab}}
\newcommand{\goto}{\rightarrow}
\newcommand{\bigo}{{\mathcal O}}
\newcommand{\half}{\frac{1}{2}}
%\newcommand\implies{\quad\Longrightarrow\quad}
\newcommand\reals{{{\rm l} \kern -.15em {\rm R} }}
\newcommand\complex{{\raisebox{.043ex}{\rule{0.07em}{1.56ex}} \hskip -.35em {\rm C}}}


% macros for matrices/vectors:

% matrix environment for vectors or matrices where elements are centered
\newenvironment{mat}{\left[\begin{array}{ccccccccccccccc}}{\end{array}\right]}
\newcommand\bcm{\begin{mat}}
\newcommand\ecm{\end{mat}}

% matrix environment for vectors or matrices where elements are right justifvied
\newenvironment{rmat}{\left[\begin{array}{rrrrrrrrrrrrr}}{\end{array}\right]}
\newcommand\brm{\begin{rmat}}
\newcommand\erm{\end{rmat}}

% for left brace and a set of choices
\newenvironment{choices}{\left\{ \begin{array}{ll}}{\end{array}\right.}
\newcommand\when{&\text{if~}}
\newcommand\otherwise{&\text{otherwise}}
% sample usage:
%  \delta_{ij} = \begin{choices} 1 \when i=j, \\ 0 \otherwise \end{choices}


% for labeling and referencing equations:
\newcommand{\eql}{\begin{equation}\label}
\newcommand{\eqn}[1]{(\ref{#1})}
% can then do
%  \eql{eqnlabel}
%  ...
%  \end{equation}
% and refer to it as equation \eqn{eqnlabel}.  


% some useful macros for finite difference methods:
\newcommand\unp{U^{n+1}}
\newcommand\unm{U^{n-1}}

% for chemical reactions:
\newcommand{\react}[1]{\stackrel{K_{#1}}{\rightarrow}}
\newcommand{\reactb}[2]{\stackrel{K_{#1}}{~\stackrel{\rightleftharpoons}
   {\scriptstyle K_{#2}}}~}

% Headers for exercises:

\newcommand{\exercise}[1]{\vskip 15pt  \noindent {\large \bf Exercise #1}%
     \nopagebreak\vskip 5pt \nopagebreak}

\newcommand{\chapexercises}[1]{%
     \cleardoublepage
     \centerline{\LARGE\bf Chapter #1 Exercises}
     \vskip .5cm
     \noindent
     From: {\it Finite Difference Methods for Ordinary and Partial 
     Differential Equations}\\  by R.~J.~LeVeque, SIAM, 2007.~~~
     {\tt http://www.amath.washington.edu/$\sim$rjl/fdmbook}
     \vskip .5cm
     }


%% Kristen's added
% Partial derivative d
\newcommand{\p}{\partial}
% integral dt
\newcommand{\dd}{\text{d}}
\newcommand{\ul}{\uvec{\ell}}
%\newcommand{\dx}{\text{dx}}
%\newcommand{\dy}{\text{dy}}
%\newcommand{\dr}{\text{dr}}
%\newcommand{\dl}{\text{d$\uvec{\ell}$}}
% Full partial derivative fraction, use like \pd[x]{t} for \frac{\partial x}{\partial t}
\newcommand{\pd[2]}{\frac{\partial #1}{\partial #2}}
% left and right big parentheses, \left{ and \right} as \lp and \rp
\newcommand{\lp}{\left(}
\newcommand{\rp}{\right)}
\newcommand{\uvec}{\underline} % for vectors
\newcommand{\uuvec[1]}{\underline{\underline #1 }} % for tensors
\newcommand{\ovec}{\overline} % for reynolds stresses

% Better tilde: http://tex.stackexchange.com/questions/9363/how-does-one-insert-a-backslash-or-a-tilde-into-latex
\newcommand{\ttilde}{{\raise.17ex\hbox{$\scriptstyle\sim$}}}
